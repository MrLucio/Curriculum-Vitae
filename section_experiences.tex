% YAAC Another Awesome CV LaTeX Template
%
% This template has been downloaded from:
% https://github.com/darwiin/yaac-another-awesome-cv
%
% Author:
% Christophe Roger
%
% Template license:
% CC BY-SA 4.0 (https://creativecommons.org/licenses/by-sa/4.0/)
%Section: Work Experience at the top

\sectionTitle{Esperienze professionali}{\faSuitcase}

\begin{experiences}
  \experience
  {Presente}    {Sviluppo Software}{Croce Rossa Italiana}{Verona}
  {Novembre 2020}
                    {
                      Sviluppo gestionale veronese per il coordinamento operativo
                    }
                    {HTML, CSS, JavaScript, PHP, MySQL}
  \emptySeparator
  \experience
  {Marzo 2021}    {Sviluppo Software}{Colombo 3000 srl}{Verona}
  {Settembre 2020}
                    {
                      Software per la creazione di file vari pronti all'invio telematico, conformi alle specifiche tecniche \mbox{dell'Agenzia delle Entrate}
                      \begin{itemize}
                        \item Sviluppo software per la compilazione automatica di modelli \emph{770} conformi alle specifiche
                        \item Sviluppo software per la compilazione automatica di modelli \emph{Certificazione Unica} conformi alle specifiche
                      \end{itemize}
                    }
                    {PHP, Java, JSON}
  \emptySeparator
  \experience
  {Ottobre 2020}      {Sviluppo Software}{Ready Information srl}{Verona}
  {Giugno 2020}
                    {
                      Sviluppo di un gestionale web
                    }
                    {Linux, Docker, Python, HTTP, Vue.js, Laravel, MongoDB} 
  \emptySeparator   
  \experience
  {Agosto 2019}       {Sviluppo Software}{Ready Information srl}{Verona}
  {Giugno 2019}
                    {
                      \begin{itemize}
                        \item Web scraping
                        \item Sviluppo di un bot \emph{Telegram} per facilitare l'interazione con i clienti
                        \item Configurazione web server tramite Python e il framework Django
                        \item Visualizzazione e interazione dei dati tramite un gestionale sviluppato con la libreria React
                      \end{itemize}
                    }
                    {Linux, Python, MySQL, Telegram, Django, JavaScript, React}
  \emptySeparator
  \experience
    {Agosto 2018} {Sviluppo Software}{NETSYSCO srl}{Verona}
    {Luglio 2018}    {
		 \github{mrlucio/Raspberry-pymodbus} \github{mrlucio/Raspberry-bacpypes}\smallskip
                      \begin{itemize}
                        \item Manutenzione di dispositivi informatici
                        \item Gestione di reti aziendali
                        \item Configurazione Samba e DHCP su server aziendali
                        \item Gestione di una rete aziendale di sensori tramite i protocolli Modbus e BACnet
                      \end{itemize}
                    }
                    {Linux, Raspbian, Samba, DHCP, Modbus, Java, Python}
  \emptySeparator
  \experience
    {Giugno 2018}   {Sviluppo Software}{I.T.I. G. Marconi}{Verona}
    {Maggio 2018} {
                      \begin{itemize}
                        \item Configurazione Raspbian O.S. su Raspberry Pi 3B
                        \item Creazione log accessi tramite badge e sensore RFID
                        \item Visualizzazione log accessi tramite browser e web server Python
                      \end{itemize}
                    }
                    {Linux, Raspbian, SQLite, RFID, Python, HTML, CSS, Bootstrap}
\end{experiences}
